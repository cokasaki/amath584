\documentclass[letter, 12pt]{article}
\usepackage[tmargin=1in,lmargin=1in,rmargin=1in,bmargin=1in,paper=letterpaper]{geometry}
\IfFileExists{minionpro.sty}
        {\usepackage[mathlf,textlf,minionint]{minionpro}}
        {\message{***Minion Pro not available***}}
% -------------------------------------------------------------
\usepackage{mathtools}
\usepackage{mathrsfs}
\usepackage{amsthm}
\usepackage{amsfonts}
\usepackage{framed}
\usepackage{enumerate}
\usepackage{ifthen}

\input{basicpreamble}
% -------------------------------------------------------------
\newif\ifpdf \ifx\pdfoutput\undefined
\pdffalse
\else
\pdftrue
\fi
\ifpdf
\usepackage[stretch=40,step=8,selected=true]{microtype}  % allow font expansion up to +/- 4% of normal width
\usepackage[pdftex]{graphicx}
\else
\usepackage[dvips]{graphicx}
\fi
% -------------------------------------------------------------
% END OF PACKAGES LOADED
\begin{document}

\parindent=0in
\newcounter{probnum}
\stepcounter{probnum}
\newenvironment{problem}[1][]
   {\begin{framed} \textbf{Problem \theprobnum: #1}}
   {\end{framed}\stepcounter{probnum}}
\newenvironment{bookproblem}[1]
   {\begin{framed} \textbf{Problem #1:}}
   {\end{framed}\stepcounter{probnum}}
%%%%%END OF HEADER

\begin{flushright}
Connie Okasaki \\
AMATH 584\\
Assignment 1\\
9 October 2020
\end{flushright}

%%=========== PROBLEM 1 ============= UNFINISHED ============%%
\begin{problem}
Show that if matrix $A$ is triangular and unitary, then it is diagonal.
\end{problem}

Suppose that $A$ is upper triangular. Since it is unitary its columns are orthogonal and have unit length. The first column must therefore be of the form $(a_{11},0,0,....,0)$ where $\norm{a_{11}}=1$. Now suppose by induction that the first $k$ columns are diagonal, i.e. each consists of a single norm-1 complex entry on the diagonal. The $k+1$st column must be orthogonal to the first $k$ columns. In order to achieve this, each inner product $\inner{a_{k+1},a_i} = a_{k+1,i}a_{ii} = 0$. Therefore, since $a_{ii}$ is non-zero $a_{k+1,i}=0$, and $a_{k+1}$ is of the same diagonal form. Therefore the whole matrix $A$ is diagonal. The same holds for lower-triangular matrices, where we induct from $a_{nn}$ instead of $a_{11}$. 

\pagebreak


%%=========== PROBLEM 2 ============= UNFINISHED ============%%
\begin{problem}
Consider that the matrices $A\in \C^{n\times m}$ and $B\in\C^{n\times m}$ are Hermitian (self-adjoint)
\begin{enumerate}[(a)]
\item Prove that all eigenvalues $\lambda_k$ of $A$ are real
\item Prove that if $x_k$ is the $k$th eigenvector, then eigenvectors with distinct eigenvalues are orthogonal
\item Prove the sum of two Hermitian matrices is Hermitian
\item Prove the inverse of an invertible Hermitian matrix is Hermitian as well
\item Prove the produce of two Hermitian matrices is Hermitian if and only if $AB=BA$.
\end{enumerate}		
\end{problem}

\begin{enumerate}[(a)]
\item By definition of the adjoint has the property $\inner{Ax_k,x_k} = \inner{x_k,A^*x_k}$. Similarly, for scalars $\inner{cx_k,x_k} = \inner{x_k,\overline{c}x_k}$. Therefore,
\[
\inner{\lambda x_k,x_k} = \inner{A x_k,x_k} = \inner{x_k,A^*x_k} = \inner{x_k,Ax_k} = \inner{x_k,\lambda x_k} = \inner{\overline{\lambda}x_k,x_k}.
\]
Therefore, $\lambda = \overline{\lambda}$ is a real number.
\item Suppose $\lambda_k \neq \lambda_i$. Since these eigenvalues are distinct at least one must be nonzero, and without loss of generality let that be $\lambda_k$. Then
\[
\inner{x_k,x_i} = \lambda_k^{-1}\inner{\lambda_kx_k,x_i} = \lambda_k^{-1}\inner{Ax_k,x_i} = \lambda_k^{-1}\inner{x_k,Ax_i} = \lambda_k^{-1}\inner{x_k,\lambda_ix_i} = \frac{\lambda_i}{\lambda_k}\inner{x_k,x_i}.
\]
If $\lambda_i=0$ then $\inner{x_k,x_i} = 0$. Otherwise, nevertheless $\frac{\lambda_i}{\lambda_k}\neq 1$ so $\inner{x_k,x_i}=0$.
\item The adjoint operation is distributive so therefore $(A+B)^* = A^* + B^* = A+B$. 
\item Suppose here that $n=m$. Then $I = (AA^{-1}) = (A^{-1})^*A^* = (A^{-1})^*A$. Therefore since the inverse of $A$ is unique, $(A^{-1})^* = A^{-1}$.
\item Suppose here that $B \in \C^{m\times ell}$. Then $(AB)^* = B^*A^* = BA$. Therefore, $(AB)^* = AB$ if and only if $BA=AB$.

\end{enumerate}

\pagebreak

%%=========== PROBLEM 3 ============= UNFINISHED ============%%
\begin{problem}
Consider the matrix $U\in \C^{n\times m}$ which is unitary
\begin{enumerate}[(a)]
\item Prove that the matrix is diagonalizable
\item Prove that the inverse is $U^{-1}=U^*$.
\item Prove it is isometric with respect to the $\ell_2$ norm, i.e. $\norm{Ux} = \norm{x}$
\item Prove that all eigenvalues have modulus unity.
\end{enumerate}
\end{problem}

We will assume here that $n=m$.
\begin{enumerate}[(a)]
\item The spectral theorem states that complex normal matrices have an orthonormal basis consisting of eigenvectors of $U$. Since $U$ is unitary, $U$ is normal, so simply choose $V$ as the transformation into the given orthonormal basis and $U = VDV^{-1}$.
\item Since the columns are orthogonal and have norm 1, $U^*U = I$. Therefore $U^{-1} = U^*$.
\item $\norm{Ux} = \inner{Ux,Ux} = \inner{x,U^*Ux} = \inner{x,x} = \norm{x}$
\item $\norm{x} = \norm{Ux} = \norm{\lambda x} = \norm{\lambda}\norm{x}$. Therefore $\norm{\lambda}=1$.
\end{enumerate}

\pagebreak

\end{document}	