\documentclass[letter, 12pt]{article}
\usepackage[tmargin=1in,lmargin=1in,rmargin=1in,bmargin=1in,paper=letterpaper]{geometry}
\IfFileExists{minionpro.sty}
        {\usepackage[mathlf,textlf,minionint]{minionpro}}
        {\message{***Minion Pro not available***}}
% -------------------------------------------------------------
\usepackage{mathtools}
\usepackage{mathrsfs}
\usepackage{amsthm}
\usepackage{amsfonts}
\usepackage{framed}
\usepackage{enumerate}
\usepackage{ifthen}

\input{basicpreamble}
% -------------------------------------------------------------
\newif\ifpdf \ifx\pdfoutput\undefined
\pdffalse
\else
\pdftrue
\fi
\ifpdf
\usepackage[stretch=40,step=8,selected=true]{microtype}  % allow font expansion up to +/- 4% of normal width
\usepackage[pdftex]{graphicx}
\else
\usepackage[dvips]{graphicx}
\fi
% -------------------------------------------------------------
% END OF PACKAGES LOADED
\begin{document}

\parindent=0in
\newcounter{probnum}
\stepcounter{probnum}
\newenvironment{problem}[1][]
   {\begin{framed} \textbf{Problem \theprobnum: #1}}
   {\end{framed}\stepcounter{probnum}}
\newenvironment{bookproblem}[1]
   {\begin{framed} \textbf{Problem #1:}}
   {\end{framed}\stepcounter{probnum}}
%%%%%END OF HEADER

\begin{flushright}
Connie Okasaki \\
AMATH 584\\
Assignment 2\\
16 Oct 2020
\end{flushright}

%%=========== PROBLEM 1 ============= UNFINISHED ============%%
\begin{problem}
Show that for a matrix $A$
\begin{enumerate}[(a)]
\item The nonzero singular values of $A$ are the square roots of the nonzero eigenvalues of $AA^*$ or $A^*A$
\item If $A=A^*$, then the singular values are the absolute values of the eigenvalues of $A$
\item Given that the determinant of a matrix $U$ is unity, show $|\det(A)|=\prod_{j=1}^m \sigma_j$
\end{enumerate}
\end{problem}

\begin{enumerate}[(a)]
\item Suppose that the SVD of $A = U\Sigma V^*$, with singular values in descending order as usual. Then:
\begin{align*}
AA^* & = U\Sigma V^* V \Sigma^* U^* \\
& = U\Sigma\Sigma^* U^* \\
AA^* U = U \Sigma\Sigma^*.
\end{align*}
This is an eigenvalue problem. Since the singular values are real, we know that $\Sigma\Sigma^* = \Sigma^2$ so that the singular values are in fact the positive square roots of the eigenvalues of $AA^*$. Likewise
\begin{align*}
A^*A & = V\Sigma U^*U\Sigma V^* \\
& = V\Sigma^2V^* \\
A^*A V & = V\Sigma^2
\end{align*}
\item If $A=A^*$ then let $V$ be the eigenvalues of $A$. We know that $AV = V \Lambda$. But then $A^*AV = V \Lambda^2 = V\Sigma^2$. Thus, since the singular values are the positive square roots of $\lambda_i^2$ we see that they are the absolute values of the eigenvalues of $A$.
\item Presumably the problem is referencing the property that $|\det(U)| = 1$ for a unitary matrix. Therefore, \[|\det(A)| = |\det(U)|\times |\det(\Sigma)| \times |\det(V)| = |\det(\Sigma)| = \prod_{j=1}^m \sigma_j.\]
\end{enumerate}

\pagebreak

\end{document}